\section{Escalando MySQL}

\subsection{Escalado vertical}
\begin{frame}{Escalado vertical}
  \begin{itemize}
    \item Ampliar o mejorar el hardware.
    \item MySQL no se adapta muy bien al escalado vertical.
    \item Depende mucho del tipo de uso que se haga del MySQL.
  \end{itemize}
\end{frame}

\subsection{Escalado horizontal}
\begin{frame}{Escalado horizontal}
  \begin{itemize}
    \item Distribución de los datos.
    \item Varias opciones:
    \begin{itemize}
      \item Replicación (y uso de los esclavos para lectura).
      \item Distribución (por clave)
      \item Distribución (por funcionalidad)
      \item Combinación de las anteriores.
    \end{itemize}
  \end{itemize}
\end{frame}

\subsection{Escalado hacia atrás}
\begin{frame}{Escalado hacia atrás}
  \begin{itemize}
    \item Eliminación de datos ya no necesarios.
    \item Datos de carácter histórico que pueden ser eliminados o migrados.
  \end{itemize}
\end{frame}

\subsection{Escalado con cluster}
\begin{frame}{Escalado con cluster}
  \begin{itemize}
    \item MySQL Cluster es una implementación de distribución de datos transparente.
    \item Distribuye los datos entre un conjunto de nodos.
    \item Da buen rendimiento para consultas simples y pocos datos.
    \item Se comporta mal con consultas complejas y que requieran comunicación entre nodos.
  \end{itemize}
\end{frame}
