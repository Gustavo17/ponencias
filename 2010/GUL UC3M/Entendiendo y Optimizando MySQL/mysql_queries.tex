\section{Las queries}

\subsection{Las queries}
\begin{frame}{Las queries}
  \begin{itemize}
    \item El cliente envia una query al servidor.
    \item Este comprueba si esta en la cache.
    \item Si esta, devuelve el resultado.
    \item Si no lo esta, pasa la query al parser.
    \item Este convierte la query en un arbol y lo pasa al optimizador.
    \item El optimizador procesa el arbol y optiene un "execution plan".
    \item Se ejecuta el execution plan sobre los S.E. que corresponda.
    \item Si es necesario se procesa el resultado en una tabla temporal.
    \item Se almacena en la cache el resultado, y se devuelve al cliente.
  \end{itemize}
\end{frame}

\subsection{Estrategia de ejecucion}
\begin{frame}{Estrategia de ejecucion}
  \begin{itemize}
    \item Toda query en mysql es un "join".
    \item Mysql crea un bucle de ejecucion recursivo.
    \item Si tiene que buscar en n tablas busca en la primera.
    \item Cuando encuentra un resultado empieza a buscar en la segunda.
    \item Asi hasta la ultima, en la cual, al encontrar resultado, devuelve la fila.
    \item Al terminar cualqueiera de las tablas, hace backtracking.
  \end{itemize}
\end{frame}

\begin{frame}{Sobre la estrategia de ejecucion}
  \begin{itemize}
    \item Algunas optimizaciones se entienden por la estrategia de mysql.
    \item Es muy importante el orden de las tablas.
    \item Es muy importante la eleccion correcta de los indices.
    \item 
    \item 
    \item 
  \end{itemize}
\end{frame}

\subsection{Analizando el rendimiento}
\begin{frame}{Errores comunes}
  \begin{itemize}
    \item Solicitar mas columnas de las que necesitas.
    \item Solicitar mas filas de las que necesitas.
    \item Solicitar datos no indexados.
  \end{itemize}
\end{frame}

\begin{frame}{Analizando el rendimiento}
  \begin{itemize}
    \item Log de consultas lentas
    \item Explain para ver el execution plan
    \item 
    \item 
  \end{itemize}
\end{frame}
