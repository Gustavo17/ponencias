\section*{Introducción}
\subsection*{Conceptos Básicos}
\begin{frame}{Términos}
  \begin{itemize}
    \item Sistema de ficheros: Conjunto de algoritmos y datos que nos permiten almacenar nuestros datos y acceder a ellos en un dispositivo físico.
    \item Sector: Unidad de almacenamiento en disco.
    \item Bloque/Sector: Conjunto de sectores contiguos que conforman la unidad de almacenamiento más pequeña de la partición.
    \item Inodo: Estructura de datos que almacena un objeto del sistema de ficheros.
    \item Sector de Arranque: Zona de la partición reservada para el arranque del sistema operativo.
    \item Superblock: Bloque de datos que contiene información referente a toda la partición.
    \item Trashing: Ineficiencia producida en un sistema de ficheros debido al excesivo movimiento de los cabezales del disco duro.
    \item Extent: Conjunto de bloques contiguos y relacionados.
  \end{itemize}
\end{frame}

\begin{frame}{Journaling}
  \begin{itemize}
    \item Sistema de seguridad de los sistemas de ficheros para preservar la consistencia.
    \item Guarda información de las operaciones para poder dar marcha atrás en caso de ser necesario.
    \item El proceso de journaling es sencillo:
    \begin{itemize}
      \item Se bloquean las estructuras afectadas.
      \item Se crea el journal.
      \item Se va apuntando y realizando las operaciones en el journal.
      \item Al terminar, si todo ha ido bien, se elimina el journal.
    \end{itemize}
    \item En los sistemas de fichero se suelen usar únicamente para los metadatos.
  \end{itemize}
\end{frame}

\begin{frame}{ACLs}
  \begin{itemize}
    \item Listas de control de acceso.
    \item Permiten un control de acceso muy granular.
    \item Se define quién accede a qué y con qué permisos.
  \end{itemize}
\end{frame}

\begin{frame}{Arboles B+}
  \begin{itemize}
    \item Árbol donde sólo los nodos hoja tiene información.
    \item Los nodos hojas están enlazados entre si (para acceso secuencial).
    \item Usado para acceso y borrado de forma eficiente.
  \end{itemize}
\end{frame}

\subsection*{Tipos de Sistemas de Ficheros}
\begin{frame}{Tipos de Sistemas de Ficheros}
  \begin{itemize}
    \item Sistemas de disco: Diseñados para almacenar ficheros en un dispositivo de almacenamiento de datos (Ej: disco duro, cdrom, dvd, discos de memoria, etc).
    \item Sistemas de ficheros de bases de datos: Están basados en metadatos, de este modo las búsquedas se realizan por consultas a dicha base de datos (Ej: Gnome VFS, BFS, WinFS,etc).
    \item Sistemas de ficheros transaccionales: Sistemas de ficheros que soportan transacciones.
    \item Sistemas de ficheros de propósito especifico: Sus contenidos son generados directamente por software (Ej: /proc, /sys, etc).
  \end{itemize}
\end{frame}
