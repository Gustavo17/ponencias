\section{Introducción}
\subsection{Conceptos Básicos}
\begin{frame}{Conceptos Básicos}
  \begin{itemize}
    \item Sistema de ficheros: Conjunto de algoritmos y datos que nos permiten almacenar nuestros datos y acceder a ellos en un dispositivo fisico.
    \item Sector: Unidad de almacenamiento en disco.
    \item Bloque/Sector: Conjunto de sectores contiguos que conforman la unidad de almacenamiento mas pequeña de la particion.
    \item Inodo: Estructura de datos que almacena un objeto del sistema de ficheros.
    \item FAT: Es una estructura que almacena informacion sobre el estado de los clusters.
    \item Sector de Arranque: Zona de la particion reservada para el arranque del sistema operativo.
    \item Superblock: Bloque de datos que contiene informacion referente a toda la partició.
    \item Trashing: Ineficiencia producida en un sistema de ficheros debido al execivo movimiento de los cabezales del disco duro.
  \end{itemize}
\end{frame}

\subsection{Tipos de Sistemas de Ficheros}
\begin{frame}{Tipos de Sistemas de Ficheros}
  \begin{itemize}
    \item Sistemas de disco: Diseñados para almacenar ficheros en un dispositivo de almacenamiento de datos (Ej: disco duro, cdrom, dvd, discos de memoria, etc).
    \item Sistemas de ficheros de bases de datos: Estan basados en metadatos, de este modo las busquedas se realizan por consultas a dicha base de datos (Ej: Gnome VFS, BFS, WinFS,etc).
    \item Sistemas de ficheros transaccionales: Sistemas de ficheros que soportan transacciones.
    \item Sistemas de ficheros de proposito especifico: Sus contenidos son generados directamente por software (Ej: /proc, /sys, etc).
  \end{itemize}
\end{frame}
