\section*{Para terminar}
\subsection*{Otra información útil}
\begin{frame}{Herramientas útiles}
  \begin{itemize}
    \item Herramientas de debug (xfs\_db, debugreiserfs, debugfs (ext2/3/4))
    \item Herramientas de benchmark: iozone, dbench, bonnie++,
    \item La documentación!!
  \end{itemize}
\end{frame}

\begin{frame}{Referencias}
  \begin{itemize}
    \item Ext2: http://e2fsprogs.sourceforge.net/ext2intro.html
    \item FAT: http://technet.microsoft.com/en-us/library/cc758586%28WS.10%29.aspx
    \item NTFS: http://technet.microsoft.com/en-us/library/cc758691%28WS.10%29.aspx
    \item ReiserFS: http://homes.cerias.purdue.edu/~florian/reiser/reiserfs.php
    \item XFS: http://oss.sgi.com/projects/xfs/training/index.html
  \end{itemize}
\end{frame}

\subsection*{Cierre}

\begin{frame}{¿Qué queda en el tintero?}
  \begin{itemize}
    \item Otros sistemas de ficheros orientados a disco (btrfs, reiser4, hfs, zfs, iso9660\dots)
    \item Sistemas de ficheros de red (NFS, AFS\dots)
    \item Sistemas de ficheros basados en metadatos (Gnome VFS, BFS, WinFS\dots)
    \item Sistemas de ficheros de propósito específico (/proc, /sys, \dots)
    \item Implementación de sistemas de ficheros (Fuse, Kernel Modules\dots)
    \item \dots
  \end{itemize}
\end{frame}

\begin{frame}{Dudas}
  \begin{center}
    \dots
  \end{center}
\end{frame}

\begin{frame}{Fin}
  \begin{center}
    Fin
  \end{center}
\end{frame}
