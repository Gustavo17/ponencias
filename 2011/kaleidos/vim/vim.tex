\documentclass[10pt]{beamer}

\usepackage[utf8]{inputenc}
\usepackage[spanish]{babel}
\usepackage{graphicx}

\mode<presentation>
\usetheme{Madrid}
%\usecolortheme[RGB={111,73,135}]{structure}
\usecolortheme[RGB={128,0,0}]{structure}
%\usecolortheme[RGB={0,96,0}]{structure}
%\usecolortheme[RGB={200,0,200}]{structure}
%\usecolortheme[RGB={0,128,0}]{structure}
%\usecolortheme[RGB={0,0,128}]{structure}
\usefonttheme{serif}
\useinnertheme{rectangles}
\useoutertheme{split}

\setbeamercovered{transparent}

\title{Introducción a Vim}
\author{Jesús Espino García}
\date{1 de Agosto de 2011}
\subject{Introducción a Vim}

\institute[Kaleidos]{Kaleidos}

\setcounter{tocdepth}{2}

\AtBeginSubsection[]
{
  \begin{frame}<beamer>{Indice}
    \tableofcontents[sectionstyle=show/shaded,subsectionstyle=show/shaded/hide]
  \end{frame}
}

\begin{document}

  \frame{\maketitle}

  \begin{frame}
    \frametitle{\begin{center}Introducción.\end{center}}
  \end{frame}

  \begin{frame}
    \frametitle{¿Que es Vim?}
    \begin{itemize}
      \item Vim ("VI IMproved").
      \item Es un editor de textos.
      \item Desarrollado por Bram Moolenaar.
      \item Basado en el vi.
      \item Licencia GPL.
    \end{itemize}
  \end{frame}

  \begin{frame}
    \frametitle{Ventajas}
  
    \begin{itemize}
      \item Es Open Source.
      \item Es muy configurable.
      \item Tiene resaltado para el codigo.
      \item Es charity-ware.
      \item No ocupa demasiado espacio.
      \item Uso muy similar al vi.
      \item Permite muchos cambios con pocas combinaciones de teclas.
      \item Comodo de usar una ves se conoce.
      \item Existen versiones para muchos sistemas operativos.
    \end{itemize}
  \end{frame}

  \begin{frame}
    \frametitle{Historia}
    \begin{itemize}
      \item 1988 - Vim 1.0 - Vi IMitation para Amiga.
      \item 1991 - Vim 1.14 - Primera versión pública.
      \item 1992 - Vim 1.22 - Portado a Unix y renombrado a Vi IMproved.
      \item 1994 - Vim 3.0 - Múltiples ventanas.
      \item 1996 - Vim 4.0 - Interfaz gráfico.
      \item 1998 - Vim 5.0 - Resaltado de sintaxsis.
      \item 2000 - Vim 6.0 - Plegado y Multilingue.
      \item 2000 - Vim 7.0 - Pestañas, corrector ortografico, omnicompletion.
    \end{itemize}
  \end{frame}

  \begin{frame}
    \frametitle{\begin{center}Conceptos Básicos.\end{center}}
  \end{frame}

  \begin{frame}
    \frametitle{Modos}
  
    Existen 3 modos:
    \begin{itemize}
      \item Comandos: Es el básico, a partir de aquí se ejecuta todo.
      \item Insertar: Es el modo que nos permite insertar texto.
      \item Visual: Este modo nos permite hacer selecciones de texto (por caracter, linea o bloques) y realizar operaciones sobre el.
    \end{itemize}
  \end{frame}

  \begin{frame}
    \frametitle{Buffers}
  
    \begin{itemize}
      \item Un contexto de edicción.
    \end{itemize}
  \end{frame}

  \begin{frame}[containsverbatim]
    \frametitle{Opciones}
  
    \begin{itemize}
      \item Vim tiene muchas opciones.
      \item set/unset: gestion de las opciones de vim.
      \item \verb+:set all+: Muestra todas las configuraciones.
      \item \verb+:set nu/nonu+: Muestra el numero de linea.
      \item \verb+:set hlsearch/nohlsearch+: Añade o quita remarcado a las búsquedas.
      \item \verb+:set filetype=<tipo>+: Especifica el tipo de fichero que se esta editando.
      \item \verb+:set textwidth=n+: Especifica el ancho del documento.
    \end{itemize}
  \end{frame}

  \begin{frame}
    \frametitle{Ayuda}
  
    \begin{itemize}
      \item :help <seccion o comando>
      \item :help intro
    \end{itemize}
  \end{frame}
  
  \begin{frame}
    \frametitle{\begin{center}Comandos Básicos.\end{center}}
  \end{frame}
  
  \begin{frame}
    Existen muchísimos comandos, pero los mas básicos serian los siguientes:
    \begin{itemize}
      \item Movimientos
      \item Edición
      \item Búsquedas
      \item Otros
    \end{itemize}
  \end{frame}
  
  \begin{frame}[containsverbatim]
    \frametitle{Movimientos}
    \begin{itemize}
      \item \verb+h j k l+: Mueve el cursor un caracter.
      \item \verb+w b e+: Mueve el cursor una palabra.
      \item \verb#0 | <n>| ^ $ - +#: Mueve el cursor por lineas.
      \item \verb+% ( ) { } [[ ]]+: Mueve el cursor por bloques.
      \item \verb+H <n>H L <n>L+: Mueve el cursor por la pantalla.
      \item \verb+^B ^U ^D ^F ^Y ^E+: Mueve la pantalla.
      \item \verb+1G <n>G G+: Movimiento del cursor por todo el fichero.
    \end{itemize}
  \end{frame}
  
  \begin{frame}[containsverbatim]
    \frametitle{Edición}
    \begin{itemize}
      \item \verb+I i a A o O r R+: Entra en modo inserción.
      \item \verb+X nX x nx dd ndd+: Borra caracteres
      \item \verb+Y yy yny+: Copia.
      \item \verb+p P+: Pega.
    \end{itemize}
  \end{frame}
  
  \begin{frame}[containsverbatim]
    \frametitle{Busquedas}
    \begin{itemize}
      \item \verb+/<cadena>+: Busca la cadena hacia adelante.
      \item \verb+?<cadena>+: Busca la cadena hacia detrás.
      \item \verb+n+: Repite la ultima búsqueda.
      \item \verb+N+: Repite la búsqueda invirtiendo el sentido.
    \end{itemize}
  \end{frame}
  
  \begin{frame}[containsverbatim]
    \frametitle{Otros}
    \begin{itemize}
      \item \verb+ZZ+: Sale y guarda el fichero.
      \item \verb+:wq+: Sale y guarda el fichero.
      \item \verb+:x+: Sale y guarda el fichero.
      \item \verb+:view <file>+: Abre un fichero para solo lectura.
      \item \verb+:w+: Escribe el fichero.
      \item \verb+:q!+: Sale del programa.
      \item \verb+u+: Deshace la última operación.
      \item \verb+^r+: Rehace el último deshacer.
      \item \verb+J+: Une dos lineas.
      \item \verb+.+: Repite la última operación realizada.
    \end{itemize}
  \end{frame}

  \begin{frame}
    \frametitle{\begin{center}Otras funcionalidades.\end{center}}
  \end{frame}
  
  \begin{frame}
    \frametitle{Otros comandos}
    \begin{itemize}
      \item Sustituciones
      \item Abreviaturas
      \item Macros
      \item Marcas
      \item Plegado
      \item Grabación de comandos
      \item Comandos del sistema
      \item Configuraciones
      \item Combinaciones
      \item Ventanas
      \item Pestañas
      \item Correción ortografica
      \item Plugins
    \end{itemize}
  \end{frame}
  
  \begin{frame}[containsverbatim]
    \frametitle{Sustituciones}
    \begin{itemize}
      \item \verb+:%s/<regexp>/<cadena>/+: Sustitulle la primera vez que encuentre la expresion regular en una linea, por la cadena.
      \item \verb+:%s/<regexp>/<cadena>/g+: Sustitulle todas las veces que encuentre la expresion regular por la cadena.
      \item \verb+:<rango>s/<regexp>/<cadena>/+: Sustituye todas las veces que encuentre la expresión regular dentro del rango, por la cadena.
    \end{itemize}
  \end{frame}
  
  \begin{frame}[containsverbatim]
    \frametitle{Abreviaturas}
    Son practicas para escritura rápida y para predefinirlas en un fichero dependiendo el documento que vayas a editar.
    \begin{itemize}
      \item \verb+:abbreviate <cadena1> <cadena2>+: Abrevia la cadena 2 como la cadena 1.
      \item \verb+:abbreviate+: Muestra las abreviaturas actuales.
    \end{itemize}
  \end{frame}
  
  \begin{frame}[containsverbatim]
    \frametitle{Macros}
    Las macros son secuencias de comandos que se asignan a un caracter.
    \begin{itemize}
      \item \verb+:map C <cadena>+: define un macro.
      \item \verb+C+: ejecuta el macro definido en C.
    \end{itemize}
    ejemplo: \verb+:map <F5> icadena de prueba <ESC>+
  \end{frame}
  
  \begin{frame}[containsverbatim]
    \frametitle{Marcas}
    Las marcas se utilizan para memorizar posiciones dentro del fichero.
    \begin{itemize}
      \item \verb+mC+: Marca una posición con el caracter C.
      \item \verb+'C+: Va a la posición marcada con C.
    \end{itemize}
    Las marcas se pueden usar también en el rango.
  \end{frame}
  
  \begin{frame}[containsverbatim]
    \frametitle{Plegado}
    El plegado se utiliza para ocultar parte del texto del archivo que no queremos ver. 
    \begin{itemize}
      \item \verb+:<rango> fold+: Pliega el rango.
      \item \verb+:<rango> folddoopen <comando>+: Ejecuta el comando sobre todo lo que no esté dentro de un pliegue cerrado.
      \item \verb+:<rango> folddoclosed <comando>+: Ejecuta el comando sobre todo lo que esté dentro de un pliegue cerrado.
      \item \verb+:<rango> foldopen[!]+: Abre los pliegues en el rango.
      \item \verb+:<rango> foldclosed[!]+: Cierra los pliegues en el rango.
      \item Para \verb+foldopen+ y \verb+foldclose+ el \verb+!+ sirve para abrir también pliegues anidados.
    \end{itemize}
  \end{frame}
  
  \begin{frame}[containsverbatim]
    \frametitle{Plegado II}
    Algunos comandos utiles para el plegado.
    \begin{itemize}
      \item \verb+zd+: Borra el pliegue sobre el que se encuentre.
      \item \verb+zD+: Borra el pliegue sobre el que se encuentre y todos los anidados.
      \item \verb+zE+: Borra todos los pliegues del archivo.
      \item \verb+zo+: Abre el pliegue sobre el que se encuentre.
      \item \verb+zO+: Abre el pliegue sobre el que se encuentre y todos los anidados.
      \item \verb+zc+: Cierra el pliegue sobre el que se encuentre.
      \item \verb+zC+: Cierra el pliegue sobre el que se encuentre y todos los anidados.
      \item \verb+za+: Cambia de estado (cerrado/abierto) del pliegue sobre el que se encuentre.
      \item \verb+zA+: Cambia de estado (cerrado/abierto) del pliegue sobre el que se encuentre y todos los anidados.
      \item \verb+zj+: Mueve el cursor hasta un pliegue más abajo.
      \item \verb+zk+: Mueve el cursor hasta un pliegue más arriba.
    \end{itemize}
  \end{frame}
  
  \begin{frame}[containsverbatim]
    \frametitle{Grabacion de comandos}
    La grabación de comandos nos permite grabar una secuencia de comandos que
    ejecutamos para repetirla posteriormente con una sencilla combinación de
    teclas.
    \begin{itemize}
      \item \verb+qC+: Empieza a grabar una sucesión de comandos que ejecutes.
      \item \verb+q+: Termina de grabar.
      \item \verb+@C+: Ejecuta de nuevo la misma sucesión de comandos.
    \end{itemize}
  \end{frame}
  
  \begin{frame}[containsverbatim]
    \frametitle{Comandos del sistema}

    \begin{itemize}
      \item \verb+:<rango>!<comando>+: Ejecuta el comando con entrada estándar
      el rango que se le ha especificado y lo sustituye por la salida del
      comando.
    \end{itemize}
  \end{frame}

  \begin{frame}[containsverbatim]
    \frametitle{Combinaciones}
    Existen algunas combinaciones simples que se usan con cierta frecuencia.
    \begin{itemize}
      \item \verb+xp+: Cambia de orden dos caracteres.
      \item \verb+ddp+: Cambia de orden dos lineas.
      \item \verb+gqq+: Reformatea una linea a 80 caracteres.
      \item \verb+gq}+: Reformatea un parrafo a 80 caracteres.
    \end{itemize}
  \end{frame}

  \begin{frame}[containsverbatim]
    \frametitle{Ventanas}
    \begin{itemize}
      \item \verb+:split+: Abre una nueva ventana.
      \item \verb+:split <file>+: Abre el fichero en una nueva ventana.
      \item \verb+:n split+: Abre una ventana de tamaño n.
      \item \verb+:sview+: Combinación entre :split y :view.
      \item \verb#^W <w|j|k|-|+|=>#: Moverse por las ventanas.
    \end{itemize}
  \end{frame}
  
  \begin{frame}[containsverbatim]
    \frametitle{Pestañas}
    \begin{itemize}
      \item \verb+:tabnew [<file>]+: Abre una nueva pestaña.
      \item \verb+:tabclose+: Cierra la pestaña.
      \item \verb+gt gT <n>gt+: Cambia de pestaña.
    \end{itemize}
  \end{frame}

  \begin{frame}[containsverbatim]
    \frametitle{Correción ortográfica}
    \begin{itemize}
      \item \verb+]s [s+: Proxima y anterior palabra mal.
      \item \verb+zg zw+: Añade y elimina palabras del diccionario.
      \item \verb+zG zW+: Añade y elimina palabras a un diccionario temporal.
      \item \verb+z=+: Sugiere palabras.
    \end{itemize}
  \end{frame}
  
  \begin{frame}[containsverbatim]
    \frametitle{Plugins}
    \begin{itemize}
      \item vim-addon-manager
      \item http://www.vim.org
    \end{itemize}
  \end{frame}
  
  \begin{frame}
    \frametitle{\begin{center}Para programadores.\end{center}}
  \end{frame}
  
  \begin{frame}[containsverbatim]
    \frametitle{Para programadores}
    \begin{itemize}
      \item Configuraciones.
      \item Make.
      \item Ctags.
    \end{itemize}
  \end{frame}
  
  \begin{frame}[containsverbatim]
    \frametitle{Configuraciones}
    \begin{itemize}
      \item \verb+:syntax on/off+: Activa o desactiva el reconocimiento de sintaxis.
      \item \verb+:set autoindent/noautoindent+: Activa o desactiva el auto indentado.
      \item \verb+:set cindent/smartindent/autoindent+: Selecciona el tipo de auto indentado que se usa.
    \end{itemize}
  \end{frame}
  
  \begin{frame}[containsverbatim]
    \frametitle{Make I}
    \begin{itemize}
      \item \verb+:make <argumentos>+: Ejecuta el make con los argumentos que se le pasen.
      \item \verb+:cnext+: Mueve el cursor a la posición del error siguiente (linea y fichero).
      \item \verb+:cprevious+: Mueve el cursor a la posición del error anterior (linea y fichero).
      \item \verb+:clast+: Mueve al ultimo error.
      \item \verb+:crewind+: Mueve al primer error.
      \item \verb+:cnfile+: Mueve al primer error del próximo fichero.
    \end{itemize}
  \end{frame}
  
  \begin{frame}[containsverbatim]
    \frametitle{Make II}
    \begin{itemize}
      \item \verb+:cc+: Muestra el error actual.
      \item \verb+:clist+: Muestra la lista de errores.
      \item \verb+:clist <rango>+: muestra el rango de errores pedido.
      \item \verb+:clist!+: muestra también los mensajes de informativos del compilador.
    \end{itemize}
  \end{frame}
  
  \begin{frame}[containsverbatim]
    \frametitle{ctags}
    búsqueda de definición de las funciones con ctags.
    \begin{itemize}
      \item \verb+^]+: va a la definición de la función.
      \item \verb+^t+: vuelve a la posición donde estaba.
    \end{itemize}
  \end{frame}

  \begin{frame}
    \frametitle{\begin{center}Para terminar.\end{center}}
  \end{frame}
 
  \begin{frame}
    \frametitle{¿Qué se queda en el tintero?}
    \begin{itemize}
      \item Scripting en vim.
      \item Scripting en vim con python.
    \end{itemize}
  \end{frame}

  \begin{frame}
    \frametitle{Referencias}
    \begin{itemize}
      \item \small{http://www.vim.org - Web ofcial de vim.}
      \item \small{http://www.truth.sk/vim/vimbook-OPL.pdf - Libro de vim.}
    \end{itemize}
  \end{frame}

  \begin{frame}
    \frametitle{Dudas}
    \dots
  \end{frame}

\end{document}
