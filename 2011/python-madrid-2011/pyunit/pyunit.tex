\documentclass[10pt]{beamer}

\usepackage[utf8]{inputenc}
\usepackage[spanish]{babel}
\usepackage{graphicx}

\mode<presentation>
\usetheme{Madrid}
%\usecolortheme[RGB={111,73,135}]{structure}
\usecolortheme[RGB={128,0,0}]{structure}
%\usecolortheme[RGB={0,96,0}]{structure}
%\usecolortheme[RGB={200,0,200}]{structure}
%\usecolortheme[RGB={0,128,0}]{structure}
%\usecolortheme[RGB={0,0,128}]{structure}
\usefonttheme{serif}
\useinnertheme{rectangles}
\useoutertheme{split}
\setbeamercovered{transparent}


\title{Pruebas unitarias con Python Unit}
\author{Jesús Espino García}
\date{15 de Julio de 2011}
\subject{Pruebas unitarias con Python Unit}

\institute[Python Madrid 2011]{Kaleidos\\Python Madrid 2011}


\setcounter{tocdepth}{2}

\AtBeginSubsection[]
{
  \begin{frame}<beamer>{Indice}
    \tableofcontents[sectionstyle=show/shaded,subsectionstyle=show/shaded/hide]
  \end{frame}
}

\begin{document}

  \frame{\maketitle}


  \section*{Introducción}
  \begin{frame}{¿Qué es Fabric?}
    \begin{itemize}
      \item Herramienta de despliegue.
      \item Herramienta ejecución de tareas.
      \item Herramienta para ejecución remota.
    \end{itemize}
  \end{frame}
  
  \begin{frame}{¿Para qué sirve Fabric?}
    \begin{itemize}
      \item Despliegue de aplicaciones en entornos complejos.
      \item Despliegue de aplicaciones en entornos distribuidos.
      \item Ejecución de tareas administrativas habituales.
      \item Ejecución de tareas administrativas en entornos distribuidos.
      \item Scripts de construcción.
      \item \dots{}
    \end{itemize}
  \end{frame}
  
  \begin{frame}{¿Por qué usar Fabric?}
    \begin{itemize}
      \item Nos evita errores.
      \item Nos evita inconsistencias.
      \item Nos facilita las tareas periódicas.
      \item Disminuye los tiempos despliegue.
      \item Es Python!!
    \end{itemize}
  \end{frame}

  \begin{frame}[containsverbatim]
    \frametitle{¿Como funciona Fabric?}
    \begin{itemize}
      \item Se escribe un fichero \verb+fabfile.py+.
      \item En este fichero se definen una serie de funciones (tareas).
      \item Se ejecuta la aplicación \verb+fab <tarea>+.
    \end{itemize}
  \end{frame}
  
  \section*{Entendiendo Fabric}
  \begin{frame}{Entorno}
    \begin{itemize}
      \item La ejecución de fabric depende del entorno definido.
      \item El entorno es un objeto diccionario.
      \item Para cambiar el comportamiento modificamos el entorno.
      \item Cambios más habituales:
      \begin{itemize}
        \item hosts: Host en los que ejecutar las tareas.
        \item user: Usuario para las conexiones SSH
        \item password: Password para las conexiones SSH
        \item warn\_only: No detener en caso de error
        \item roledefs: Definición de roles (grupos de hosts)
        \item roles: Roles sobre los que ejecutar las tareas.
        \item fabfile: Fichero a ejecutar en vez de fabfile.py.
      \end{itemize}
    \end{itemize}
  \end{frame}

  \begin{frame}{fabricrc}
    \begin{itemize}
      \item El fichero $\sim$/.fabricrc nos permite configuraciones generales.
      \item Este fichero afecta a todas las ejecuciones de fabric.
      \item Contiene un serie de parejas clave/valor.
      \item Las parejas clave valor rellenan el entorno.
    \end{itemize}
  \end{frame}

  \begin{frame}{Modelo de ejecución}
    \begin{itemize}
      \item Construcción de una lista de tareas.
      \item Construcción de una lista de hosts.
      \item Se recorre la lista de tareas.
      \item Para cada hosts se ejecuta la tarea actual.
    \end{itemize}
  \end{frame}

  \begin{frame}[containsverbatim]
    \frametitle{Lista de tareas}
    \begin{itemize}
      \item Fabric extrae las tareas del fichero \verb+fabfile.py+ (Normalmente).
      \item Considera tarea todo objeto ''Callable'' que haya al importar el \verb+fabfile.py+.
      \item No se consideran tareas las funciones/métodos del propio fabric.
      \item No se consideran tareas los objetos que su nombre empieza por \_.
      \item Se pueden listar las tareas ejecutando \verb+fab --list+.
    \end{itemize}
  \end{frame}

  \begin{frame}{Lista de hosts}
    \begin{itemize}
      \item La lista de hosts se extrae del entorno.
      \item Si la tarea tiene un decorador de hosts sobreescribe los hosts del entorno.
      \item Si se le pasa como parámetro los hosts sobreescribe los hosts del entorno y decorador.
      \item Se extrae la lista de roles del entorno.
      \item Si la tarea tiene un decorador de roles se sobreescriben los roles del entorno.
      \item Si se le pasa como parámetro los roles sobreescribe los roles del entorno y decorador.
      \item Se hace la union de los hosts y de los hosts contenidos en los roles.
    \end{itemize}
  \end{frame}

  \begin{frame}[containsverbatim]
    \frametitle{Tareas con parámetros}
    \begin{itemize}
      \item Las tareas pueden recibir parámetros.
      \item El paso de parámetros sigue el siguiente modelos \verb+fab <tarea>:<parm1>,<parm2>,<parm_name>=<parm_value>+
      \item El parámetro hosts permite definir los hosts sobre los que ejecutar.
      \item Si se encuentra el parámetro hosts se extrae de la lista de parámetros de la tarea.
    \end{itemize}
  \end{frame}
  
  \subsection*{La API de Fabric}
  \begin{frame}{Core}
    \begin{itemize}
      \item colors (blue, cyan, green, magenta, red, white, yellow)
      \item context\_managers (cd, hide, lcd, path, prefix, settings, show)
      \item decorators (hosts, roles, runs\_once)
      \item network (disconnect\_all)
      \item operations (get, open\_shell, put, reboot, run, sudo, local, prompt, require)
      \item utils (abort, fastprint, indent, puts, warn)
    \end{itemize}
  \end{frame}

  \begin{frame}{Contrib}
    \begin{itemize}
      \item contrib.console (confirm)
      \item contrib.django (project, settings\_module)
      \item contrib.files (append, comment, contains, exists, first, sed, uncomment, upload\_template)
      \item contrib.project (rsync\_project, upload\_project)
    \end{itemize}
  \end{frame}
  
  \subsection*{Ejemplos}
  \begin{frame}{Ejemplos}
    \begin{itemize}
      \item Despliegue de aplicación web distribuida.
      \item Despliegue de otra aplicación web distribuida.
      \item Despliegue de aplicación web compleja.
      \item Constructor de software.
      \item Script de administración.
    \end{itemize}
  \end{frame}

  \section*{Para terminar}

  \begin{frame}{Dudas}
    \begin{center}
      \dots
    \end{center}
  \end{frame}

\end{document}
