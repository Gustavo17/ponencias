\documentclass[10pt]{beamer}

\usepackage[utf8]{inputenc}
\usepackage[spanish]{babel}
\usepackage{graphicx}

\mode<presentation>
\usetheme{Madrid}
%\usecolortheme[RGB={111,73,135}]{structure}
\usecolortheme[RGB={128,0,0}]{structure}
%\usecolortheme[RGB={0,96,0}]{structure}
%\usecolortheme[RGB={200,0,200}]{structure}
%\usecolortheme[RGB={0,128,0}]{structure}
%\usecolortheme[RGB={0,0,128}]{structure}
\usefonttheme{serif}
\useinnertheme{rectangles}
\useoutertheme{split}

\setbeamercovered{transparent}

\title{Introducción a IPython}
\author{Jesús Espino García}
\date{15 de Julio de 2011}
\subject{Introducción a IPython}

\institute[Python Madrid 2011]{Kaleidos\\Python Madrid 2011}

\setcounter{tocdepth}{2}

\AtBeginSubsection[]
{
  \begin{frame}<beamer>{Indice}
    \tableofcontents[sectionstyle=show/shaded,subsectionstyle=show/shaded/hide]
  \end{frame}
}

\begin{document}

  \frame{\maketitle}


  \section*{Introducción}
  \begin{frame}{¿Qué es IPython?}
    \begin{itemize}
      \item Interprete de python con esteroides.
      \item Entorno pensado para la programación interactiva.
      \item Entorno pensado para la programación exploratoria.
    \end{itemize}
  \end{frame}
  
  \begin{frame}{¿Para qué sirve IPython?}
    \begin{itemize}
      \item Depuración.
      \item Profiling.
      \item Micro prototipos.
      \item Pruebas de bibliotecas.
      \item \dots{}
    \end{itemize}
  \end{frame}
  
  \section*{Principales funcionalidades}
  \begin{frame}{Tab completion}
    \begin{itemize}
      \item Autocompletado con tabulador.
      \item Lista los atributos de un objeto.
      \item Tambien autocompleta nombres de fichero.
    \end{itemize}
  \end{frame}

  \begin{frame}{Exploración de objetos}
    \begin{itemize}
      \item IPython define macros para la exploración de objetos.
      \item \%pdoc muestra el docstring de un objeto.
      \item \%pdef muestra la definición de un objeto (metodo, clase).
      \item \%psource muestra la implementación de un objeto (metodo, clase).
      \item \%pfile muestra el codigo de todo el fichero donde se define un objeto (metodo, clase).
      \item \%psearch busca objetos que corresponda con un patron.
      \item obj? o obj?? muestra la informacion, o mas información sobre un objeto.
      \item \%pdef? nos mostraria la información sobre pdef.
    \end{itemize}
  \end{frame}

  \begin{frame}{Trabajando con archivos}
    \begin{itemize}
      \item IPython define una macro \%run para ejecutar el contenido de un fichero.
      \item Cada \%run lee y ejecuta todo el fichero (no es un import).
      \item El comando \%run tiene 3 flags importantes.
      \begin{itemize}
        \item -t mide el tiempo de ejecucion.
        \item -d ejecuta el codigo bajo pdb (en realidad ipdb).
        \item -p ejecuta el codigo bajo un profiler.
      \end{itemize}
      \item \%pycat nos muestra el contenido de un fichero con resaltado de sintaxis
    \end{itemize}
  \end{frame}

  \begin{frame}{Usandolo como una shell}
    \begin{itemize}
      \item IPython tiene autocompletado de nombres de ficheros y directorios.
      \item Cada \%run lee y ejecuta todo el fichero (no es un import).
      \item Tambien tiene algunos comandos utiles tipicos de una shell
      \begin{itemize}
        \item \%alias/\%unalias define un alias para un comando
        \item \%ls lista los ficheros
        \item \%cd me mueve de directorio
        \item \%pwd me devuelve el directorio actual
        \item \%env me devuelve el entorno actual
        \item \%pushd me mueve a un directorio y guarda el anterior en una pila
        \item \%popd saca un directorio de la pila y me mueve a el
        \item \%dirs me muestra la pila de directorios
        \item \%dhist historico de directorios
        \item !command ejecuta system(command)
        \item !!command ejecuta el comando y me devuelve un objeto SList
        \item \%bg ejecuta una funcion en segundo plano
        \item \%bookmark añadir directorios a un lista de favoritos
      \end{itemize}
    \end{itemize}
  \end{frame}

  \begin{frame}
    \frametitle{Profiling}
    \begin{itemize}
      \item \%time ejecuta una funcion y dice el tiempo que ha tardado en ejecutar
      \item \%prun ejecuta una funcion a traves del profiler
      \item \%timeit ejecuta una funcion N veces y dice el tiempo que ha tardado en ejecutar
    \end{itemize}
  \end{frame}
  
  \begin{frame}{Input/Output caches}
    Cache de Salida
    \begin{itemize}
      \item Todo resultado de ejecucción es guardado en la cache de salida Out.
      \item El objeto Out es una lista donde el id es el numero de linea que se ejecuto.
      \item Tambien existe un alias \_N ara cada entrada en la cache de salida.
      \item Tambien existen tres alias \_, \_\_ y \_\_\_  que se refieren a las tres ultimas lineas ejecutadas.
    \end{itemize}

    Cache de Entrada
    \begin{itemize}
      \item Todo ejecucion de lineas se almacena en la cache de entrada In
      \item Es una lista de strings.
      \item Si queremos volver a ejecutar una serie de lineas podemos usar ''exec In[23:27]+In[5:11]''
      \item Tambien existe un alias \_iN ara cada entrada en la cache de salida.
      \item Tambien existen tres alias \_i, \_ii y \_iii  que se refieren a las tres ultimas lineas ejecutadas.
    \end{itemize}
  \end{frame}

  \begin{frame}{Macros}
    IPython nos permite definir macros que luego podamos ejecutar y guardar.
    \begin{itemize}
      \item \%macro Define una nueva macro con un nombre y unas lineas de In.
      \item \%edit Nos permite editar macros ya creadas (y mucho mas).
      \item \%store Nos permite dar persistencia a una macro (y a otras variables).
    \end{itemize}
  \end{frame}

  \begin{frame}
    \frametitle{Perfiles}
    \begin{itemize}
      \item Existe una configuración de IPython para el usuario.
      \item IPython permite definir perfiles de configuración para tareas concretas.
      \item Por ejemplo podemos definir un perfil para matematicas, o interaccion con bbdd.
    \end{itemize}
  \end{frame}

  \section*{Para terminar}
  
  \begin{frame}{Referencias}
    \begin{itemize}
      \item Proyecto ipython: http://ipython.org.
      \item Documentación de ipython: http://ipython.org/ipython-doc/stable/html/index.html.
      \item Introduccion a IPython: ? (En el interprete de ipython).
      \item Repositorio Git: https://github.com/ipython
      \item Python-ES: python-es@python.org
    \end{itemize}
  \end{frame}

  \begin{frame}{Dudas}
    \begin{center}
      \dots
    \end{center}
  \end{frame}

\end{document}
