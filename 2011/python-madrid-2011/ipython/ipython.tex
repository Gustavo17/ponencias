\documentclass[10pt]{beamer}

\usepackage[utf8]{inputenc}
\usepackage[spanish]{babel}
\usepackage{graphicx}

\mode<presentation>
\usetheme{Madrid}
%\usecolortheme[RGB={111,73,135}]{structure}
\usecolortheme[RGB={128,0,0}]{structure}
%\usecolortheme[RGB={0,96,0}]{structure}
%\usecolortheme[RGB={200,0,200}]{structure}
%\usecolortheme[RGB={0,128,0}]{structure}
%\usecolortheme[RGB={0,0,128}]{structure}
\usefonttheme{serif}
\useinnertheme{rectangles}
\useoutertheme{split}

\setbeamercovered{transparent}

\title{Introducción a IPython}
\author{Jesús Espino García}
\date{15 de Julio de 2011}
\subject{Introducción a IPython}

\institute[Python Madrid 2011]{Kaleidos\\Python Madrid 2011}

\setcounter{tocdepth}{2}

\AtBeginSubsection[]
{
  \begin{frame}<beamer>{Indice}
    \tableofcontents[sectionstyle=show/shaded,subsectionstyle=show/shaded/hide]
  \end{frame}
}

\begin{document}

  \frame{\maketitle}


  \section*{Introducción}
  \begin{frame}{¿Qué es IPython?}
    \begin{itemize}
      \item Interprete de python con esteroides.
      \item Entorno pensado para la programación interactiva.
      \item Entorno pensado para la programación exploratoria.
    \end{itemize}
  \end{frame}
  
  \begin{frame}{¿Para qué sirve IPython?}
    \begin{itemize}
      \item Depuración.
      \item Profiling.
      \item Micro prototipos.
      \item Pruebas de bibliotecas.
      \item \dots{}
    \end{itemize}
  \end{frame}
  
  \section*{Principales funcionalidades}
  \begin{frame}{Tab completion}
    \begin{itemize}
      \item Autocompletado con tabulador.
      \item Lista los atributos de un objeto.
      \item Tambien autocompleta nombres de fichero.
    \end{itemize}
  \end{frame}

  \begin{frame}{Exploración de objetos}
    \begin{itemize}
      \item IPython define macros para la exploración de objetos.
      \item \%pdoc muestra el docstring de un objeto.
      \item \%pdef muestra la definición de un objeto (metodo, clase).
      \item \%psource muestra la implementación de un objeto (metodo, clase).
      \item \%pfile muestra el codigo de todo el fichero donde se define un objeto (metodo, clase).
    \end{itemize}
  \end{frame}

  \begin{frame}{Ejecución de archivos}
    \begin{itemize}
      \item IPython define una macro \%run para ejecutar el contenido de un fichero.
      \item Cada \%run lee y ejecuta todo el fichero (no es un import).
      \item El comando \%run tiene 3 flags importantes.
      \begin{itemize}
        \item -t mide el tiempo de ejecucion.
        \item -d ejecuta el codigo bajo pdb (en realidad ipdb).
        \item -p ejecuta el codigo bajo un profiler.
      \end{itemize}
    \end{itemize}
  \end{frame}

  \begin{frame}{Input/Output caches}
    \begin{itemize}
      \item 
    \end{itemize}
  \end{frame}

  \begin{frame}
    \frametitle{Perfiles}
    \begin{itemize}
      \item 
    \end{itemize}
  \end{frame}
  
  \begin{frame}{Macros}
    \begin{itemize}
      \item 
    \end{itemize}
  \end{frame}

  \section*{Para terminar}

  \begin{frame}{Dudas}
    \begin{center}
      \dots
    \end{center}
  \end{frame}

\end{document}
