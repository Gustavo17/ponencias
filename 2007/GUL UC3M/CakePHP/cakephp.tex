\documentclass{beamer}
\usepackage[spanish]{babel}
\usepackage[latin1]{inputenc}

%\usepackage{beamerouterthemeinfolines}
%\usepackage{beamerthemeBoadilla}      % Sin {,sub}apartados, blanco
%\usepackage{beamerthemeRochester}     % Sin {,sub}apartados, simple, cuadrado
%\usepackage{beamerthemeboxes}         % Sin {,sub}apartados, muy blanco
%\usepackage{beamerthemePittsburgh}    % Sin {,sub}apartados, muy blanco, redondo
\usepackage{beamerthemeMadrid}        % Sin {,sub}apartados
%\usepackage{beamerthemeSingapore}     % Sin subapartados, blanco, puntitos
%\usepackage{beamerthemeCopenhagen}    % Violeta/negro, cargado, con todo
%\usepackage{beamerthemeLuebeck}       % Como Copenhagen, cuadrado
%\usepackage{beamerthemeWarsaw}        % Violeta/negro, gradiente, cargado, con todo

\title{CakePHP: MVC para PHP}
\author{Jes�s Espino Garc�a\\ \texttt{jespinog@gmail.com}}
\date{8 de Noviembre del 2007}

\pgfdeclareimage[interpolate=true,%
                 width=11cm]{diagrama}{img/diagrama}

\pgfdeclareimage[interpolate=true,%
                 width=2cm]{cakephp}{img/cakephp}

\institute[UC3M]{\pgfuseimage{cakephp}}

\begin{document}

\frame{\titlepage}

\section<presentation>*{�ndice}

\frame
{
  \frametitle{�ndice}
  \tableofcontents[part=1]
}

\AtBeginSubsection[]
{
  \frame<handout:0>
  {
    \frametitle{�ndice}
    \tableofcontents[current,currentsubsection]
  }
}


\part<presentation>{Contents}

\section{Introducci�n}

\subsection{�Qu� es?}


%%%%%%%%%%%%%%%%%%%%%%%%%%%%%%%%%%%%%%%%%%%%%%%%%%%%%%%%%%%%%%%%%% �Qu� es? %
\frame
{
  \frametitle{�Qu� es?}
  \begin{columns}
    \begin{column}{7cm}
      \begin{itemize}
        \item Framework de desarrollo web.
        \item<2-> Inspirado en Ruby on Rails.
        \item<3-> Aparece en el 2005 cuando esta en auge RoR.
        \item<4-> Sistema MVC.
      \end{itemize}
    \end{column}
    \begin{column}{4cm}
      \uncover<4->{\pgfuseimage{cakephp}}
    \end{column}
  \end{columns}
}

\subsection{�Por qu� usarlo?}

%%%%%%%%%%%%%%%%%%%%%%%%%%%%%%%%%%%%%%%%%%%%%%%%%%%%%%%%%%%%% �Por qu� usarlo? %
\frame
{
  \frametitle{�Por qu� usarlo?}
  \begin{itemize}
    \item Es PHP $:-)$
    \item Simplifica y agiliza el desarrollo.
    \item Aumenta la escalabilidad.
    \item Organiza mejor nuestro codigo.
    \item A�ade de manera sencilla caracteristicas complejas (validacion, autenticacion, ajax...).
    \item Esta bien documentado.
  \end{itemize}
}


\subsection{�Por qu� no usarlo?}

%%%%%%%%%%%%%%%%%%%%%%%%%%%%%%%%%%%%%%%%%%%%%%%%%%%%%%%%%%%%% �Por qu� no usarlo? %
\frame
{
  \frametitle{�Por qu� no usarlo?}
  \begin{itemize}
    \item Es PHP $:-($
    \item Existen cosas mas potentes (RoR).
    \item Sentimiento de perdida de control.
    \item Orientado a sistemas con BBDD.
  \end{itemize}
}

\subsection{�Como funciona?}

%%%%%%%%%%%%%%%%%%%%%%%%%%%%%%%%%%%%%%%%%%%%%%%%%%%%%%%%%%%%% �Como funciona? %
\frame
{
  \frametitle{�Como funciona?}
  \begin{itemize}
    \item Arquitectura MVC:
    \begin{itemize}
      \item Modelo: Acceso a los datos (BBDD).
      \item Vista: HTML que se muestra al usuario (Plantillas).
      \item Controlador: Logica de programa (PHP).
    \end{itemize}
  \end{itemize}
}

\frame
{
  \frametitle{�Como funciona?}
  \begin{center}
    \pgfuseimage{diagrama}
  \end{center}
}

\section{Trabajando con Cake}

\subsection{Scaffolding}

%%%%%%%%%%%%%%%%%%%%%%%%%%%%%%%%%%%%%%%%%%%%%%%%%%%%%%%%%%%%% Scaffolding %
\frame
{
  \frametitle{Scaffolding}
  \begin{itemize}
    \item Generacion automatica de codigo.
    \item Util y practico, pero no definitivo.
  \end{itemize}
}

\frame[containsverbatim]
{
  \frametitle{Ejemplo}
  \begin{itemize}
    \item BBDD
    \begin{verbatim}
CREATE DATABASE charlacake;
CREATE TABLE to_dos(id INT NOT NULL PRIMARY KEY AUTO_INCREMENT, title VARCHAR(100), description VARCHAR(255), finished BOOLEAN);
    \end{verbatim}
    \item Controlador
    \begin{verbatim}
<?php
class ToDoController extends AppController
{
    var $scaffold;
}
?>
    \end{verbatim}
    \item Modelo
    \begin{verbatim}
<?php
class ToDo extends AppModel
{
    var $name = 'ToDo';

    var $displayField = 'title';
}
?>
    \end{verbatim}
  \end{itemize}
}

\subsection{El modelo}
%%%%%%%%%%%%%%%%%%%%%%%%%%%%%%%%%%%%%%%%%%%%%%%%%%%%%%%%%%%%% El modelo %
\frame
{
  \frametitle{El modelo}
  \begin{itemize}
    \item Convierte las tablas en objetos PHP.
    \item Simplifica el acceso a los datos (Buscar, modificar, crear...).
    \item Define las relaciones entre tablas (1:1, 1:N, N:M).
  \end{itemize}
}

\frame[containsverbatim]
{
  \frametitle{Ejemplo}
  \begin{verbatim}
<?php
class ToDo extends AppModel
{
    var $name = 'ToDo';
}
?>
  \end{verbatim}
}

\subsection{El controlador}
%%%%%%%%%%%%%%%%%%%%%%%%%%%%%%%%%%%%%%%%%%%%%%%%%%%%%%%%%%%%% El controlador %
\frame
{
  \frametitle{El controlador}
  \begin{itemize}
    \item Define las acciones de nuestra aplicacion (borrar, editar, ver...).
    \item Nos permite dirijir el flujo del programa (Redirecciones, Flashes...).
    \item Acceso a la informacion (Cookies, sesiones, formularios...).
  \end{itemize}
}

\frame[containsverbatim]
{
  \frametitle{Ejemplo}
  \begin{verbatim}
<?php
class ToDoController extends AppController
{
    var $name = 'ToDo';

    function index()
    {
        $this->set('todos', $this->ToDo->findAll());
    }
}
?>
  \end{verbatim}
}

\subsection{La vista}
%%%%%%%%%%%%%%%%%%%%%%%%%%%%%%%%%%%%%%%%%%%%%%%%%%%%%%%%%%%%% La vista %
\frame
{
  \frametitle{La vista}
  \begin{itemize}
    \item Plantillas de codigo HTML con PHP empotrado.
    \item El PHP accede a los datos pasados por el controlador.
    \item El resultado es la aplicacion de los datos a la plantilla.
  \end{itemize}
}

\frame[containsverbatim]
{
  \frametitle{Ejemplo}
  \begin{verbatim}
<h1>Tareas pendientes</h1>
<table>
    <tr>
        <th>Id</th>
        <th>Titulo</th>
        <th>Descripcion</th>
        <th>Hecha</th>
    </tr>

    <?php foreach ($todos as $todo): ?>
    <tr>
        <td><?php echo $todo['ToDo']['id']; ?></td>
        <td><?php echo $html->link($todo['ToDo']['title'], "/todos/view/".$todo['ToDo']['id']); ?></td>
        <td><?php echo $todo['ToDo']['description']; ?></td>
        <td><?php if($todo['ToDo']['finished']){echo "si";}else{echo "no";} ?></td>
    </tr>
    <?php endforeach; ?>

</table>
  \end{verbatim}
}

\section{Para terminar}

%%%%%%%%%%%%%%%%%%%%%%%%%%%%%%%%%%%%%%%%%%%%%%%%%%%%%%%%%%%%% �Que se nos queda en el tintero? %
\subsection{�Qu� se nos queda en el tintero?}

\frame
{
  \frametitle{�Qu� se nos queda en el tintero?}
  \begin{itemize}
    \item Validacion de campos en formularios. 
    \item Gestion de Sessiones y Cookies.
    \item Convenciones.
    \item Componente de seguridad.
  \end{itemize}
}

%%%%%%%%%%%%%%%%%%%%%%%%%%%%%%%%%%%%%%%%%%%%%%%%%%%%%%%%%%%%% Referencias %
\subsection{Referencias}

\frame
{
  \frametitle{Referencias}
  \begin{itemize}
    \item http://www.cakephp.org: Pagina oficial de CakePHP.
    \item http://www.cakeforge.org: Sistema de gestion de proyectos CakePHP.
    \item http://bakery.cakephp.org: Documentacion variada sobre CakePHP.
    \item http://manual.cakephp.org: Documentacion principal de CakePHP.
  \end{itemize}
}

%%%%%%%%%%%%%%%%%%%%%%%%%%%%%%%%%%%%%%%%%%%%%%%%%%%%%%%%%%%%% Dudas %
\subsection{Dudas}
\frame
{
  \frametitle{Dudas}
  \begin{center}
    \dots
  \end{center}
}


\end{document}
