\section{Subsistema de Memoria}
\subsection{Direcciones de memoria}
\begin{frame}{Tipos de Direcciones}
	\begin{itemize}
		\item Dirección lógica: Dirección del segmento mas el offset.
		\item Dirección lineal (o virtual): Dirección en toda la memoria.
		\item Dirección física: Dirección de acceso a la RAM.
	\end{itemize}
\end{frame}

\subsubsection{Segmentos}
\begin{frame}{¿Qué es un segmento?}
	\begin{itemize}
		\item Conjunto de memoria.
		\item Se utiliza para establecer unos permisos a la memoria.
		\item Se accede a sus contenidos a través de dirección del segmento + offset.
		\item Esta implementado en hardware.
	\end{itemize}
\end{frame}

\begin{frame}{¿Cómo lo usa Linux?}
	\begin{itemize}
		\item Lo usa de un modo muy limitado.
		\item Se prefiere el uso de páginas.
		\item Linux crea 4 segmentos que abarca toda la memoria (cada uno).
		\begin{itemize}
			\item usercode
			\item userdata
			\item kernelcode
			\item kerneldata
		\end{itemize}
		\item Como todos los segmentos empiezan en 0, la posición se define únicamente por el offset.
		\item Linux solo usa segmentos cuando lo requiere la arquitectura.
	\end{itemize}
\end{frame}

\subsubsection{Paginas}
\begin{frame}{¿Qué son las páginas?}
	\begin{itemize}
		\item Conjunto de memoria.
		\item Se utiliza para establecer unos permisos a la memoria.
		\item Se accede a su contenido a través de una dirección compuesta de varios elementos.
		\item Los elementos de la dirección designan posiciones en unas tablas de índices.
		\item Esta implementado en hardware.
	\end{itemize}
\end{frame}

\begin{frame}{¿Cómo las usa Linux?}
	\begin{itemize}
		\item Se usa un modelo con 4 tablas de direccionamiento (para poder albergar mucha memoria):
		\begin{itemize}
			\item Page Global Directory
			\item Page Upper Directory
			\item Page Middle Directory
			\item Page Table
		\end{itemize}
		\item Se usa el mismo modelo para 32 y 64 bits.
		\item Cuando se usa 32 bits y no se usa PAE, Linux elimina las tablas Upper y Middle.
		\item Cuando se usa 32 bits y se usa PAE, Linux elimina la tabla Upper.
	\end{itemize}
\end{frame}

\subsection{Gestión de memoria}
\subsubsection{Buddy System}
\begin{frame}{Buddy System}
	\begin{itemize}
		\item Sistema encargado de reservar las páginas y hacer un seguimiento de las páginas libres.
		\item Soluciona el problema de la fragmentación externa.
		\item Se almacenan las páginas en 11 listas dependiendo del grupo de páginas contiguas disponibles.
		\item Linux utiliza varios Buddy System, uno para cada ''zona'' de memoria (DMA, normal, high).
	\end{itemize}
\end{frame}
\subsubsection{Slab Allocator}
\begin{frame}{Slab Allocator}
	\begin{itemize}
		\item El Buddy System reserva memoria como mínimo del tamaño de la página.
		\item El Slab Allocator permite reservar bloques de tamaños arbitrarios mas pequeños que una página.
		\item Funciona encima del Buddy System.
		\item Tiene un sistema de caches que permite reutilizar bloques de memoria.
		\item La reutilización mejora el rendimiento de Linux que utiliza los slab con frecuencia.
	\end{itemize}
\end{frame}

\subsubsection{Reclamado de páginas}
\begin{frame}{Reclamado de páginas}
	\begin{itemize}
		\item Al hacer una petición al Buddy System y no encontrar páginas apropiadas, este hace un reclamado.
		\item El sistema de reclamado divide las páginas en 4 tipos:
		\begin{itemize}
			\item Unreclaimable: Paginas que no pueden o no es necesario reclamar.
			\item Swappable: Paginas que pueden pasarse a la swap.
			\item Syncable: Paginas que pueden ser sincronizadas a disco.
			\item Discardables: Paginas que pueden ser descargadas.
		\end{itemize}
		\item Dependiendo del tipo de páginas hace una acción u otra.
		\item El reclamado de páginas también se ejecuta periódicamente mediante kswapd.
	\end{itemize}
\end{frame}
