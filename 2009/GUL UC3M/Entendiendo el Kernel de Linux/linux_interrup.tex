\section{Interrupciones}
\subsection{Tipos interrupciones}
\begin{frame}{Tipos de interrupciones}
	\begin{itemize}
		\item Excepciones: Interrupciones del propio procesador.
		\item Interrupciones: Señales (normalmente por IRQ) del hardware.
	\end{itemize}
\end{frame}

\subsection{IRQs y APIC}
\begin{frame}{IRQs y APIC}
	\begin{block}{IRQ}
		\begin{itemize}
			\item IRQ es una señal especial que va directamente a la CPU.
			\item Tiene su propio canal de comunicación (normalmente incluido en el bus del sistema).
			\item Este bus permite mandar señales al procesador en cualquier momento.
		\end{itemize}
	\end{block}
	\begin{block}{APCI}
		\begin{itemize}
			\item APIC es un sistema de IRQs en 2 niveles (normalmente) para sistemas SMP.
			\item Permite discriminar el envío de IRQs a un procesador u otro.
			\item Cada CPU tiene su sistema de APIC local para gestionar sus interrupciones.
		\end{itemize}
	\end{block}
\end{frame}

\subsection{Excepciones}
\begin{frame}{Excepciones}
	\begin{itemize}
		\item Indica un error emitido por la CPU.
		\item Normalmente es una situación anómala producida por la programación.
		\item Ocurre durante la ejecución del proceso y no hay cambio de contexto.
	\end{itemize}
\end{frame}

\subsection{Interrupciones}
\begin{frame}{Interrupciones}
	\begin{itemize}
		\item Indica una solicitud al procesador de que se haga algo.
		\item Normalmente las interrupciones las producen el hardware.
		\item Puede ser urgente o no.
		\item En caso de ser urgente se interrumpe la ejecución y lanza el manejador.
		\item En caso de no ser urgente se espera al momento idóneo para lanzar el manejador.
		\item Para manejar una interrupción se produce un cambio de contexto.
		\item Se pueden emitir interrupciones por software (softirq).
		\item Las interrupciones pueden ser interrumpidas por otras.
	\end{itemize}
\end{frame}
