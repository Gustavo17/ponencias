\section{Subsistema de procesos}
\subsection{Los procesos}
\subsubsection{¿Qué es un proceso?}
\begin{frame}{¿Qué es un proceso?}
	\begin{itemize}
		\item Estructura de datos en el SO.
		\item Proceso completo o hilo.
		\item Instancia de tarea a realizar por la CPU.
		\item Utiliza, reserva y comparte recursos.
	\end{itemize}
\end{frame}

\subsubsection{Tipos de procesos}
\begin{frame}{Tipos de procesos}
	\begin{itemize}
		\item Procesos: Instancia de un programa en ejecución.
		\item Procesos ligeros: Instancia del ''contador de programa''.
	\end{itemize}
\end{frame}

\subsubsection{Descriptor de proceso}
\begin{frame}{Descriptor de proceso}
	\begin{block}{Información que contiene}
		\begin{itemize}
			\item Estado.
			\item Memoria asignadas.
			\item Ficheros abiertos.
			\item Directorio actual.
			\item Padre.
			\item Señales recibidas.
			\item y mucho más.
		\end{itemize}
	\end{block}
\end{frame}

\subsubsection{Estados}
\begin{frame}{Estados de un proceso}
	\begin{itemize}
		\item TASK\_RUNNING: Proceso en la CPU o a espera de CPU.
		\item TASK\_INTERRUPTIBLE: Proceso durmiendo a espera de algo.
		\item TASK\_UNINTERRUPTIBLE: Como TASK\_INTERRUPTIBLE, pero sin poder ser interrumpido.
		\item TASK\_STOPPED: Se ha parado la ejecución del proceso.
		\item TASK\_TRACED: Se ha parado la ejecución del proceso por un debugger.
		\item TASK\_ZOMBIE: Se ha terminado el proceso pero no hay ningún wait esperándole.
		\item TASK\_DEAD: El proceso esta siendo eliminado definitivamente.
	\end{itemize}
\end{frame}

\subsubsection{Cambio de proceso}
\begin{frame}{Cambio de proceso}
	\begin{itemize}
		\item Muchos cambios de proceso por segundo.
		\item Actividad necesariamente rápida.
		\item Se deben a las interrupciones.
		\item La interrupción por tiempo permite hacer planificación de procesos.
		\item Consiste en almacenar el estado del proceso actual y sustituirlo por otro.
		\item El kernel (casi en su totalidad) es ''reentrant''.
	\end{itemize}
\end{frame}

\subsubsection{Crear procesos}
\begin{frame}{Crear procesos}
	\begin{block}{Procesos}
		\begin{itemize}
			\item Proceso 0: El swapper o idle, se crea al inicio de manera ''artesanal''.
			\item Proceso 1: El init, se crea como copia del proceso 0 y ejecuta el programa init (ya en modo usuario).
			\item Resto de los procesos: Se crean como copia del proceso padre (normalmente init).
		\end{itemize}
	\end{block}
	\begin{block}{Estrategias}
		\begin{itemize}
			\item Copy On Write: Padre e hijo comparten memoria para lectura y en el momento de la escritura se reserva una nueva pagina.
			\item Procesos ligeros: Padre e hijo comparte memoria.
		\end{itemize}
	\end{block}
\end{frame}

\subsubsection{Destruir procesos}
\begin{frame}{Destruir procesos}
	\begin{itemize}
		\item Recibe la syscall \_exit().
		\item Pasa del estado en que este a TASK\_ZOMBIE.
		\item Su padre hace un wait, y pasa al estado TASK\_DEAD.
		\item Se eliminan todos los datos del proceso.
		\item Se elimina su descriptor del sistema.
		\item Los compiladores normalmente incluyen \_exit() de manera automática al final del binario.
	\end{itemize}
\end{frame}

\subsubsection{Hilos del kernel}
\begin{frame}{Hilos del kernel}
	\begin{itemize}
		\item Son procesos creados dentro del kernel.
		\item Algunos se crean al inicio.
		\item Otros bajo demanda.
		\item Algunos ejemplos son:
		\begin{itemize}
			\item kapmd: Gestiona eventos relacionados con APM.
			\item kswapd: Reclama memoria periódicamente.
			\item pdflush: Vacía los buffers ''sucios'' a disco.
		\end{itemize}
	\end{itemize}
\end{frame}

\subsection{Planificación (Scheduling)}
\begin{frame}{Algoritmos}
	\begin{block}{O(1)}
		\begin{itemize}
			\item Scheduler hasta el 2.6.22
			\item Basado en listas de prioridad.
			\item Calculo del quamto basado en prioridad estática.
			\item Calculo del sucesor basado en prioridad dinámica.
			\item La prioridad estática viene dada por el tipo de proceso (normal o tiempo real) y el nice.
			\item La prioridad dinámica se calcula a partir de la prioridad estática y el ''average sleep'' del proceso.
		\end{itemize}
	\end{block}
\end{frame}

\begin{frame}{Algoritmos}
	\begin{block}{CFS (Completely Fair Scheduler)}
		\begin{itemize}
			\item Scheduler desde 2.6.23
			\item Mas simple
			\item Reparte el tiempo del procesador de manera equitativa.
			\item Puntúa mas a los procesos que han usado menos el procesador.
			\item Penaliza a los que más.
			\item Intenta que el reparto sea justo.
		\end{itemize}
	\end{block}
\end{frame}

\subsection{Memoria de procesos}
\begin{frame}{Memoria de procesos}
	\begin{itemize}
		\item Un proceso usar un conjunto de direcciones (regiones de memoria).
		\item Cada proceso cuenta con una pila que puede usar para albergar memoria dinámica.
		\item Cada región de memoria tiene unos permisos de acceso para el proceso.
		\item Varios procesos pueden acceder a las mismas posiciones de memoria (bibliotecas, shared memory\dots).
		\item El intento de acceso a cualquier otra posición de memoria genera una excepción.
		\item El intento de uso no debido de las posiciones de memoria asignadas genera una excepción.
		\item Al acceder a su memoria puede ser que no este la pagina (por diversos motivos), por lo cual se produce una demanda de pagina.
	\end{itemize}
\end{frame}
