\section{Entendiendo Linux}
\subsection{Gestión de procesos}
\begin{frame}{¿Qué es un proceso?}
	\begin{itemize}
		\item Estructura de datos en el SO.
		\item Proceso completo o hilo.
		\item Instancia de tarea a realizar por la CPU.
		\item Utiliza, reserva y comparte recursos.
	\end{itemize}
\end{frame}

\begin{frame}{Prioridades}
	\begin{itemize}
		\item Los procesos tienen 2 prioridades.
		\item Estática (Asignada por el tipo de tarea y el nice)
		\item Dinámica (Calculada por el scheduler)
		\item El nivel nice determina el tamaño del quamto del proceso.
	\end{itemize}
\end{frame}

\begin{frame}{Context Switch}
	\begin{itemize}
		\item Para simular multitarea se va alternando la ejecución de procesos.
		\item Cada cambio de contexto tiene unos costes de tiempo.
	\end{itemize}
\end{frame}

\begin{frame}{Interrupciones}
	\begin{itemize}
		\item El hardware y el software produce interrupciones.
		\item Las interrupciones, normalmente producen cambios de contexto.
		\item Las interrupciones hardware, por lo general, se envían a todos los procesadores.
	\end{itemize}
\end{frame}

\begin{frame}{Scheduler}
	\begin{itemize}
		\item Es el encargado de seleccionar un proceso u otro.
		\item Antes del 2.6.23 el scheduler era el O(1).
		\item En el 2.6.23 y posteriores el scheduler es CFS.
	\end{itemize}
\end{frame}

\subsection{Gestión de memoria}
\begin{frame}{Memoria principal}
	\begin{itemize}
		\item Segmentos: Linux apenas los utiliza.
		\item Paginas: Lo que utiliza normalmente Linux.
		\item En sistemas de 32bits se debe dividir la memoria en 2 partes:
		\begin{itemize}
			\item Una accesible directamente (hasta los 896M).
			\item Y el resto mapeada a través de la zona accesible directamente (hasta 64G usando PAE).
		\end{itemize}
	\end{itemize}
\end{frame}

\begin{frame}{Sistema de caches}
	\begin{itemize}
		\item Los sistemas modernos tienen caches de memoria.
		\item Las caches son memorias mas rápidas que evitan el acceso a la memoria principal
		\item Usan ciertas políticas y algoritmos para determinar de que se guarda cache.
	\end{itemize}
\end{frame}

\begin{frame}{Swap}
	\begin{itemize}
		\item Memoria basada en dispositivos de disco.
		\item Almacena paginas de memoria a las que no se acceden en mucho tiempo.
		\item En caso de falta de falta de memoria principal se utiliza la swap como tal.
	\end{itemize}
\end{frame}

\begin{frame}{Asignación de Memoria}
	\begin{itemize}
		\item Al solicitar memoria, si hay paginas libres las asigna.
		\item Si no hay paginas libres, utiliza métodos de reclamado de pagina.
		\item Para esto utiliza el Buddy System, que se encarga de:
		\begin{itemize}
			\item Mantener una lista de las paginas libres.
			\item Solicita la liberación paginas a kswapd.
			\item Evitar la fragmentación de la memoria.
		\end{itemize}
		\item kswapd libera paginas basándose en ciertas políticas.
	\end{itemize}
\end{frame}

\subsection{Gestión de IO}
\begin{frame}{Sistemas de ficheros}
	\begin{itemize}
		\item VFS: Capa de abstracción para el acceso a sistemas de ficheros.
		\item Journaling: Log de las operaciones realizadas.
		\item Sistema de Ficheros: Estructura de datos para almacenar ficheros (normalmente en disco).
	\end{itemize}
\end{frame}

\begin{frame}{Acceso a disco (IO)}
	\begin{itemize}
		\item Los sistemas de ficheros acceden a disco a través del sistema IO.
		\item El sistema IO organiza los acceso a discos con el IO Scheduler.
		\item Y este finalmente escribe los datos a través del driver de disco.
	\end{itemize}
\end{frame}

\begin{frame}{IO Schedulers}
	\begin{itemize}
		\item Se pueden escoger entre 4:
		\begin{itemize}
			\item Anticipatory: Intenta guardar datos para escribir en tramas largas y continuas. Además usa deadline.
			\item CFQ (Completely Fair Queuing): Implementa un reparto equitativo del IO entre procesos.
			\item Deadline: Round robin con deadline.
			\item NOOP: No hace ninguna operación de orden. Es un FIFO.
		\end{itemize}
	\end{itemize}
\end{frame}

\subsection{Métricas de rendimiento}
\begin{frame}{Métricas de procesador}
	\begin{itemize}
		\item Uso de CPU.
		\item User time: Tiempo utilizado por los procesos de usuario (no kernel).
		\item System time: Tiempo utilizado por operaciones del kernel.
		\item Waiting: Tiempo esperando al subsistema IO.
		\item Idle time: Tiempo esperando tareas.
		\item Nice time: Tiempo utilizado para restablecer prioridades de los procesos.
		\item Load average: Carga del equipo (no es un porcentaje).
		\item Runable processes: Numero de procesos listos para correr.
		\item Blocked: Procesos esperando IO.
		\item Context Switch: Cantidad de cambios de contexto.
		\item Interrupts: Cantidad de interrupciones
	\end{itemize}
\end{frame}

\begin{frame}{Metricas de memoria}
	\begin{itemize}
		\item Memoria libre.
		\item Uso de Swap.
		\item Buffer y cache: Caches asignadas al sistema de ficheros o a dispositivos de bloque. 
		\item Slabs: Uso de memoria del kernel (no swapeable).
		\item Memoria activa/inactiva: La memoria inactiva es la que es candidata de ser swapeada.
	\end{itemize}
\end{frame}

\begin{frame}{Metricas de red}
	\begin{itemize}
		\item Paquetes recibidos y enviados.
		\item Bytes recibidos y enviados.
		\item Colisiones por segundo.
		\item Paquetes "tirados".
		\item Overruns: Veces que la tarjeta de red se ha quedado sin espacio en el buffer.
		\item Errors: Numero de tramas marcadas como fallidas.
	\end{itemize}
\end{frame}

\begin{frame}{Metricas de IO}
	\begin{itemize}
		\item iowait: Tiempo de CPU gastado esperando al subsistema IO.
		\item Media de longitud de la cola: Media de peticiones IO en cola.
		\item Media de espera: Tiempo medio de respuesta de una petición IO.
		\item Transferencias por segundo: Numero de operaciones IO por segundo.
		\item Bloques por segundo r/w
		\item KB por segundo r/w
	\end{itemize}
\end{frame}
