\documentclass{beamer}
\usepackage[spanish]{babel}
\usepackage[latin1]{inputenc}

%\usepackage{beamerouterthemeinfolines}
%\usepackage{beamerthemeBoadilla}      % Sin {,sub}apartados, blanco
%\usepackage{beamerthemeRochester}     % Sin {,sub}apartados, simple, cuadrado
%\usepackage{beamerthemeboxes}         % Sin {,sub}apartados, muy blanco
%\usepackage{beamerthemePittsburgh}    % Sin {,sub}apartados, muy blanco, redondo
%\usepackage{beamerthemeMadrid}        % Sin {,sub}apartados
%\usepackage{beamerthemeSingapore}     % Sin subapartados, blanco, puntitos
\usepackage{beamerthemeCopenhagen}    % Violeta/negro, cargado, con todo
%\usepackage{beamerthemeLuebeck}       % Como Copenhagen, cuadrado
%\usepackage{beamerthemeWarsaw}        % Violeta/negro, gradiente, cargado, con todo

% Image declarations
% \pgfdeclaremask{imagemask}{imagemaskfile}
% \pgfdeclareimage[interpolate=true,mask=imagemask,%
                 % width=2cm,height=1.3cm]{imagename}{imagename}
\pgfdeclareimage[interpolate=true,%
                 width=8cm]{notshareddata}{notshareddata}
\pgfdeclareimage[interpolate=true,%
                 width=8cm]{shareddata}{shareddata}
\pgfdeclareimage[interpolate=true,%
                 width=8cm]{drbd}{drbd}
\pgfdeclareimage[interpolate=true,%
                 width=8cm]{failover}{failover}
\pgfdeclareimage[interpolate=true,%
                 width=8cm]{gui}{gui}
\pgfdeclareimage[interpolate=true,%
                 width=5cm]{gul}{gul}
\pgfdeclareimage[interpolate=true,%
                 width=10cm]{drbd}{drbd}

\title{LinuxHA+DRBD: Alta disponibilidad de bajo coste}
\author{Jes�s Espino Garc�a\\ \texttt{jespinog@gmail.com}}
\institute{\pgfuseimage{gul}}

\date{}

\begin{document}

\frame{\titlepage}

%%%%%%%%%%%%%%%%%%%%%%%%%%%%%%%%%%%%%%%%%%%%%%%%%%%%%%%%%%%%
\frame
{
  \frametitle{�Qu� es clustering?}
  \begin{itemize}
    \item Conjunto de maquinas.
    \item Trabajando en equipo.
    \item Con un mismo objetivo.
    \item Los casos mas habituales son:
    \begin{itemize}
      \item Cluster computacional.
      \item Alta Disponibilidad.
      \item Balanceo de carga.
    \end{itemize}
  \end{itemize}
 

}

%%%%%%%%%%%%%%%%%%%%%%%%%%%%%%%%%%%%%%%%%%%%%%%%%%%%%%%%%%%
\frame
{
  \frametitle{Alta Disponibilidad VS Toleraciona a Fallos}
  \begin{itemize}
    \item Tolerancia a fallos:
    \begin{itemize}
      \item Caro.
      \item Hardware tolerante a fallos (redundancia interna, detecci�n de errores).
      \item Perdidas de servicio muy breves, entre d�cimas de segundo y unos pocos segundos.
    \end{itemize}
    \item Alta disponibilidad:
    \begin{itemize}
      \item Barato.
      \item Redundancia de hardware.
      \item Perdida de servicios entre algunos segundos hasta unos pocos minutos.
      \item Hace uso en subsistemas tolerantes a fallos.
    \end{itemize}
  \end{itemize}
}

%%%%%%%%%%%%%%%%%%%%%%%%%%%%%%%%%%%%%%%%%%%%%%%%%%%%%%%%%%%%
\frame
{
  \frametitle{�Qu� es Linux HA?}
  \begin{itemize}
    \item Un sistema de clustering de Alta disponibilidad.
    \item Libre.
    \item Maduro: En el mercado desde 1998.
    \item Extendido: Mas de 30.000 cluster en producci�n.
    \item Una comunidad de desarrollo abierta liderada por IBM y Novell.
    \item Distribuido con la mayor�a de las distribuciones.
    \item Sin requerimientos de hardware especial.
    \item Independiente del kernel (todo en espacio de usuario).
  \end{itemize}
}

%%%%%%%%%%%%%%%%%%%%%%%%%%%%%%%%%%%%%%%%%%%%%%
\begin{frame}
  \frametitle{�Qu� nos permite hacer?}
  \begin{itemize}
    \item Elaborar sistemas resistentes a un �nico punto de fallo.
    \item Los tiempos de ca�da son muy cortos.
    \item Aproximadamente a�ade un ''9'' a tu disponibilidad.
    \item Hasta 16 nodos.
    \item Recuperaci�n en caso de fallo de nodo o de servicio.
    \item Recuperaci�n en caso de perdida de conectividad.
    \item Clusters activo-pasivo y activo-activo.
    \item Monitorizaci�n de recursos.
    \item Interfaz gr�fico de configuraci�n y monitorizaci�n.
  \end{itemize}
\end{frame}


%%%%%%%%%%%%%%%%%%%%%%%%%%%%%%%%%%%%%%%%%%%%%%%
\begin{frame}
  \frametitle{�Qu� no puede hacer?}
  \begin{itemize}
    \item Dar el 100\% de disponibilidad (nadie puede).
  \end{itemize}
\end{frame}

%%%%%%%%%%%%%%%%%%%%%%%%%%%%%%%%%%%%%%%%%%%%%%
\frame{
  \frametitle{Algunos numeros}
  \begin{itemize}
    \item $99.9999\%$ $\to{}$ 30sec
    \item $99.999\%$ $\to{}$ 5min
    \item $99.99\%$ $\to{}$ 52min
    \item $99.9\%$ $\to{}$ 9h
    \item $99\%$ $\to{}$ 3.5d�as
  \end{itemize}
}


%%%%%%%%%%%%%%%%%%%%%%%%%%%%%%%%%%%%%%%%%%%%%%%
\begin{frame}[fragile]
  \frametitle{Algunos conceptos}
  \begin{itemize}
    \item Nodo: Maquina (real o virtual) que forma parte del cluster.
    \item Recurso: Algo que manejar (Servicio, IP, Unidad de disco...).
    \item Agente de recurso: Un (init) script que controla un recurso.
    \item Heartbeat: Pulso de comunicaci�n que indica a un nodo el estado de los dem�s.
    \item Failover: Toma de control por parte de un nodo de los servicios de otro.
    \item Split Brain: Falla la comunicaci�n ''Heartbeat'' entre dos grupos diferentes de nodos.
    \item Quorum: Sistema de acuerdos en el cluster.
    \item Fencing: Exclusion de un nodo del cluster.
    \item SPOFs (Single Point of Failure).
  \end{itemize}
\end{frame}

%%%%%%%%%%%%%%%%%%%%%%%%%%%%%%%%%%%%%%%%%%%%%%%%%%%%%%%%%%%%%%%%%%%
\frame
{
  \frametitle{Failover}
  \pgfuseimage{failover}
}

%%%%%%%%%%%%%%%%%%%%%%%%%%%%%%%%%%%%%%%%%%%%%%%%%%%%%%%%%%%%%%%%%%%
\frame
{
  \frametitle{Split Brain}
  \begin{itemize}
    \item Divisi�n del cluster por culpa de fallos.
    \item Normalmente de comunicaci�n del Heartbeat.
    \item Cada nodo crea ser el nodo activo.
    \item OJO!! Peligro de corrupci�n de datos.
  \end{itemize}
}

%%%%%%%%%%%%%%%%%%%%%%%%%%%%%%%%%%%%%%%%%%%%%%%%%%%%%%%%%%%%%%%%%%
\begin{frame}
  \frametitle{Quorum}
  \begin{itemize}
    \item Es un intento de evitar la mayor parte de los fallos de split brain.
    \item Trata de asegurase que solo uno de las divisiones este activa.
    \item Uno de los sistemas mas comunes de quorum es el voto. Solo la divisi�n con mas nodos puede ejecutar el cluster.
    \begin{itemize}
      \item Esto no funciona muy bien para 2 nodos.
    \end{itemize}
    \item Otro sistema de quorum es el acceso a un recurso exclusivo, el que llegue despu�s, desiste y cede los servicios.
  \end{itemize}
\end{frame}

%%%%%%%%%%%%%%%%%%%%%%%%%%%%%%%%%%%%%%%%%%%%%%%%%%%%%%%%%%%%%%%%
\begin{frame}
  \frametitle{Fencing}
  \begin{itemize}
    \item Excluye a los nodos que est�n dando fallos para que no accedan a los recursos del cluster.
    \item Linux HA soporta:
    \begin{itemize}
      \item STONITH (Shoot The Other Node In The Head) para node Fencing.
      \item Sistemas Self-Fencing (Sistemas que aseguran acceso exclusivo a un recurso (ServerRAID de IBM)) para resource fencing.
    \end{itemize}
  \end{itemize}
\end{frame}


%%%%%%%%%%%%%%%%%%%%%%%%%%%%%%%%%%%%%%%%%%%%%%%%%%%%%%%%%%%%%%%
\begin{frame}
  \frametitle{SPOFs}
  \begin{itemize}
    \item Elemento del sistema que en caso de fallo puede dejar todo el sistema inutilizado.
    \item Un buen dise�o elimina todos los SPOFs.
    \item Existe un limite, normalmente impuesto por el coste, en la eliminaci�n de SPOFs.
    \item Mucho cuidado con los SPOFs no evidentes:
    \begin{itemize}
      \item Unidades raid con una sola controladora.
      \item Enlaces redundantes usando el mismo cable.
    \end{itemize}
  \end{itemize}
\end{frame}


%%%%%%%%%%%%%%%%%%%%%%%%%%%%%%%%%%%%%%%%%%%%%%%%%%%%%%%%%%%%%%
\begin{frame}
  \frametitle{Sistemas NON-SPOFs}
  \begin{itemize}
    \item Importante las tres R de la alta disponibilidad:
    \begin{itemize}
      \item<2-> Redundancia.
      \item<3-> Redundancia.
      \item<4-> Redundancia.
    \end{itemize}
    \item<5-> La mayor�a de los SPOFs se eliminan con redundancia.
    \item<6-> Siempre existen alg�n punto de fallo simple, hay que saber donde parar.
  \end{itemize}
\end{frame}

%%%%%%%%%%%%%%%%%%%%%%%%%%%%%%%%%%%%%%%%%%%%%%%%%%%%%%%%%%%
\begin{frame}
  \frametitle{Compartiendo datos: Mediante unidad compartida}
  \begin{itemize}
    \item El modo normal.
    \item Eleva el precio de cluster.
    \item Da muy buen rendimiento.
    \item Soluci�n de bajo coste (AoE).
  \end{itemize}
\end{frame}

%%%%%%%%%%%%%%%%%%%%%%%%%%%%%%%%%%%%%%%%%%%%%%%%%%%%%%%%%%%
\begin{frame}
  \frametitle{Compartiendo datos: Mediante unidad compartida}
  \pgfuseimage{shareddata}
\end{frame}

%%%%%%%%%%%%%%%%%%%%%%%%%%%%%%%%%%%%%%%%%%%%%%%%%%%%%%%%%%%
\begin{frame}
  \frametitle{Compartiendo datos: Replicaci�n de los datos}
  \begin{itemize}
    \item El modo barato.
    \item No necesitas unidad compartida.
    \item El rendimiento es menor.
    \item Se basa en los sistemas de replicaci�n de los propios servicios y en servicios de replicaci�n de ficheros (DRBD).
    \item Una relaci�n calidad/precio excelente.
  \end{itemize}
\end{frame}

%%%%%%%%%%%%%%%%%%%%%%%%%%%%%%%%%%%%%%%%%%%%%%%%%%%%%%%%%%%
\begin{frame}
  \frametitle{Compartiendo datos: Replicaci�n de los datos}
  \pgfuseimage{drbd}
\end{frame}

%%%%%%%%%%%%%%%%%%%%%%%%%%%%%%%%%%%%%%%%%%%%%%%%%%%%%%%%%%%
\begin{frame}
  \frametitle{Compartiendo datos: Sin datos que compartir}
  \begin{itemize}
    \item Es raro, pero existen casos.
    \item Firewalls.
    \item Balanceadores de carga.
    \item Servidores Proxy.
    \item Servidores Web est�ticos (replicaci�n inicial de los contenidos).
  \end{itemize}
\end{frame}

%%%%%%%%%%%%%%%%%%%%%%%%%%%%%%%%%%%%%%%%%%%%%%%%%%%%%%%%%%%
\begin{frame}
  \frametitle{Compartiendo datos: Sin datos que compartir}
  \pgfuseimage{notshareddata}
\end{frame}

%%%%%%%%%%%%%%%%%%%%%%%%%%%%%%%%%%%%%%%%%%%%%%%%%%%%%%%%%
\begin{frame}
  \frametitle{Sistemas de ficheros de Cluster}
  \begin{itemize}
    \item Preparados para accesos concurrentes desde varios nodos.
    \item Pensados para mantener la consistencia de los datos.
    \item Los sistemas de ficheros para cluster libres mas conocidos son:
    \begin{itemize}
      \item GFS: Global File System (Comprado por RedHat y licenciado como GPL).
      \item OCFS2: Oracle Cluster File System (Desarrollado por Oracle, GPL).
    \end{itemize}
  \end{itemize}
\end{frame}

%%%%%%%%%%%%%%%%%%%%%%%%%%%%%%%%%%%%%%%%%%%%%%%%%%%%%%%%%
\begin{frame}
  \frametitle{Interfaz gr�fica}
  \pgfuseimage{gui}
\end{frame}

%%%%%%%%%%%%%%%%%%%%%%%%%%%%%%%%%%%%%%%%%%%%%%%%%%%%%%%%%%%%
\frame
{
  \frametitle{�Qu� es DRBD?}
  \begin{itemize}
    \item Servicio de replica de dispositivos.
    \item GPL.
    \item Funcionamiento sobre dispositivos fisicos o virtuales.
    \item Funcionamiento bajo dispositivos virtuales (LVM).
    \item Sincronizacion Activo-Pasivo y Activo-Activo.
    \item Modos Asincrono, Semi-Sincrono y Sincrono.
  \end{itemize}
}

%%%%%%%%%%%%%%%%%%%%%%%%%%%%%%%%%%%%%%%%%%%%%%%%%%%%%%%%%%%%
\frame
{
  \frametitle{�Como funciona DRBD?}
  \pgfuseimage{drbd}
}

%%%%%%%%%%%%%%%%%%%%%%%%%%%%%%%%%%%%%%%%%%%%%%%%%%%%%%%%%%%%
\frame
{
  \frametitle{Ejemplo en vivo}
  \begin{itemize}
    \item Montar dispositivo con sincronizacion DRBD.
    \item Montar PeaceMaker.
    \item Montar servicios sobre DRBD y PeaceMaker.
  \end{itemize}
}

%%%%%%%%%%%%%%%%%%%%%%%%%%%%%%%%%%%%%%%%%%%%%%%%%%%%%%%%%
\begin{frame}
  \frametitle{Referencias}
  \begin{itemize}
    \item Linux-HA: http://www.linux-ha.org
    \item DRBD: http://www.drbd.org
    \item Comunidad: http://linux-ha.org/ContactUs (Listas, IRC, bugtracking...)
    \item GFS: http://sources.redhat.com/cluster/gfs/
    \item OCFS2: http://oss.oracle.com/projects/ocfs/
  \end{itemize}
\end{frame}

%%%%%%%%%%%%%%%%%%%%%%%%%%%%%%%%%%%%%%%%%%%%%%%%%%%%%%%%%
\begin{frame}
  \frametitle{Dudas y preguntas}
  \center{Dudas\dots{}}
\end{frame}
\end{document}
